\documentclass{article}
\usepackage[english]{babel}
\usepackage{amsfonts}
%\usepackage{a4wide}
\usepackage{setspace}
\usepackage{amsmath}
\usepackage{amsfonts}
\usepackage{amssymb,enumerate}
\usepackage{amsthm,stmaryrd}
\usepackage[all]{xy}
\usepackage{hyperref}

\newcommand{\llc}{\mathcal{C}}


\textwidth 6.0in \oddsidemargin 0.0in

\begin{document}

\begin{center}

\textbf{Homework 2, MATH 9010}

Due on Thursday, September 8\\

Shuai Wei

\end{center}

\vspace{3 mm}

\noindent \textbf{Problem 1} Suppose $\mathcal{C}$ is a collection of subsets of $\Omega$ that satisfies the following properties:  (i) $\Omega \in \mathcal{C}$; (ii) if $A \in \mathcal{C}$, then $A^{c} \in \mathcal{C}$; (iii) If $\{A_{i}\}_{i = 1}^{\infty}$ is a collection of disjoint subsets in $\mathcal{C}$, then $\cup_{i=1}^{\infty}A_{i} \in \mathcal{C}$.

\vspace{2 mm}

Show that $\mathcal{C}$ is a $\lambda$-system.


\begin{proof}
	We will show $\llc$ satisfies the definition of $\lambda$-system.
	\begin{enumerate}[(1)]
		\item $\Omega \in \llc$ by property (i).
		\item If $A, B \in \llc$ and $A \subset B$, then $B\setminus A = B \cap A^c = (B^c \sqcup A)^c$. \\
			$B^c \in \llc$ by property (ii), and then $(B^c \sqcup A) \subset \llc$ by property (iii) since $B^c \cap A = \emptyset$. \\
		So $B\setminus A = (B^c \sqcup A)^c \in \llc$. 
		\item 
		Let $A_n \in \llc, n\geq 1$ and $A_1 \subset A_2 \subset A_3 \subset ...$.\\
		Define $B_1 = A_1 \in \llc$ and $B_n = A_n \setminus A_{n-1} $ for $n \geq 2$.\\
		Then $B_n \in \llc$ by (2) and $B_j \cap B_k = \emptyset$.\\
		So $\bigcup_{n=1}^{\infty}A_n = \bigsqcup_{n=1}^{\infty}B_n \in \llc$ by property (iii).
	\end{enumerate}
	Thus, $\llc$ is a $\lambda$-system.
\end{proof}

\end{document}
