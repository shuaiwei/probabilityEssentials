\documentclass{article}
\usepackage[english]{babel}
\usepackage{amsfonts}
%\usepackage{a4wide}
\usepackage{setspace}
\usepackage{amsmath}
\usepackage{amsfonts}
\usepackage{amssymb,enumerate}
\usepackage{amsthm,stmaryrd}
\usepackage[all]{xy}
\usepackage{hyperref}
\usepackage{flexisym}

\newcommand{\lls}{\mathcal{S}}
\newcommand{\bbp}{\mathbb{P}}
\newcommand{\bbn}{\mathbb{N}}
\newcommand{\lld}{\mathcal{D}}
\newcommand{\lla}{\mathcal{A}}
\newcommand{\llc}{\mathcal{C}}

\textwidth 6.0in \oddsidemargin 0.0in

\begin{document}
 \setcounter{page}{6}
\vspace{3 mm}

\noindent \textbf{Problem 4} (Jacod and Protter, page 45 Problem 7.14) Let $\{A_{k}\}_{k \geq 1}$ be a sequence of null events, i.e. events where $\mathbb{P}(A_{k}) = 0$ for     each $k \geq 1$.  Show that $\cup_{k=1}^{\infty}A_{k}$ is also a null event.

\begin{proof}
 	$ $\newline
	Let $\Omega$ be sample space and $\lla$ be the $\sigma$-algebra containg $\{A_{k}\}_{k \geq 1}$.\\
	Since $\{A_{k}\}_{k \geq 1}$ be a sequence of null events, \\
	there exists $\{B_{k}\}_{k \geq 1} \subset \lla$ such that $A_k \subset B_k$ and $\bbp(B_k) = 0$ for $k \geq 1$.\\
	Then $\bigcup_{k=1}^{\infty} A_k \subset \bigcup_{k=1}^{\infty} B_k \in \lla$.\\
	Let $C_1 = B_1, C_k = B_k\setminus \bigcup_{i=1}^{k-1}B_i \in \lla, k\geq 2$, then\\
	$C_k\textprime s $ are disjoint and $\bbp(C_k) \leq \bbp(B_k) $by the monotonicity of probability measure.\\
	So $\bbp({\bigcup_{k=1}^{\infty} B_k}) = \bbp({\bigsqcup_{k=1}^{\infty} C_k}) = \sum_{k=1}^{\infty}\bbp(C_k) \leq \sum_{k=1}^{\infty}\bbp(B_k) = \sum_{k=1}^{\infty}0 = 0$.\\ 
	Since we find $\bigcup_{k=1}^{\infty} B_k$ such that $\bigcup_{k=1}^{\infty} A_k \subset \bigcup_{k=1}^{\infty} B_k$ and $\bbp({\bigcup_{k=1}^{\infty} B_k}) = 0$,\\
	$\bigcup_{k=1}^{\infty}A_{k}$ is also a null event.
\end{proof}
\end{document}
