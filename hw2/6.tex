\documentclass{article}
\usepackage[english]{babel}
\usepackage{amsfonts}
%\usepackage{a4wide}
\usepackage{setspace}
\usepackage{amsmath}
\usepackage{amsfonts}
\usepackage{amssymb,enumerate}
\usepackage{amsthm,stmaryrd}
\usepackage[all]{xy}
\usepackage{hyperref}

\newcommand{\lls}{\mathcal{S}}
\newcommand{\bbp}{\mathbb{P}}
\newcommand{\bbn}{\mathbb{N}}
\newcommand{\lld}{\mathcal{D}}
\newcommand{\lla}{\mathcal{A}}
\newcommand{\llc}{\mathcal{C}}
\newcommand{\llf}{\mathcal{F}}

\textwidth 6.0in \oddsidemargin 0.0in

\begin{document}

\vspace{3 mm}



\noindent \textbf{Problem 5} (Resnick, pg. 63 Problem 1) Let $\Omega$ be a nonempty set, and let $\mathcal{F}_{0}$ be the collection of all subsets such that either $A$ or $A^{c}$ is finite.

\vspace{2 mm}

\noindent Define, for each $A \in \mathcal{F}_{0}$, the set function $\mathbb{P}$, where
\begin{eqnarray*} \mathbb{P}(A) = \left\{
                                    \begin{array}{ll}
                                      0, & \hbox{if $A$ is finite;} \\
                                      1, & \hbox{if $A^{c}$ is finite.}
                                    \end{array}
                                  \right. \end{eqnarray*}

\vspace{2 mm}

\noindent (a) If $\Omega$ is countably infinite, show $\mathbb{P}$ is additive on $\mathcal{F}_{0}$, but not countably additive.

\vspace{2 mm}

\noindent (b) If $\Omega$ is uncountable, show $\mathbb{P}$ is countably additive on $\mathcal{F}_{0}$.



\begin{proof} 
	$ $\newline
	Let $\{A_k\}_{k\geq 1} \subset \llf_0$ be pairwise disjoint.\\
	We claim there is at most one $A_k$ in $\{A_k\}_{k=1}^{\infty}$ satisfying that $A_k^c$ is finite.\\
	Suppose there are disjoint sets $A_k$ and $A_j$ satisfying that both of $A_k^c$ and $A_j^c$ are finite.\\
	Then $A_k \cap A_j = \emptyset$ and 
	\begin{equation}\label{111} 
		\bbp(A_k \cap A_j) = 0.\\
	\end{equation}
	We also have $(A_k)^c \cup (A_j)^c = (A_k \cap A_j)^c $ is finite by assumption.\\
	So $\bbp (A_k \cap A_j) = 1$, which is contradicted by (\ref{111}).\\
	Thus, there is at most one $A_k$ in $\{A_k\}_{k=1}^n$ satisfying that $A_k^c$ is finite.

\begin{enumerate}[(a)]
	\item
		\begin{enumerate}[(1)]
		\item
		Consider finite subsets $\{A_k\}_{k=1}^n$ of $\llf_0$.
		\begin{enumerate}[(i)]
			\item If $A_k$ is finite for $1 \leq k \leq n$ , $\cup_{k=1}^n A_k$ is finite.\\
				Then $\bbp(A_k) = 0$, for $1 \leq k\leq n$ and $\bbp(\cup_{k=1}^n A_k) = 0.$\\
				So $\sum_{k=1}^n \bbp (A_k) = 0 = \bbp(\cup_{k=1}^n A_k)$.
			\item
			Assume there exists the unique $i \in \bbn, 1\leq i \leq n$ such that $A_i^c$ is finite.\\
			Then $\bbp(A_i) = 1$ and $\bbp(A_j) = 0$ for $j \neq i, 1 \leq j \leq n$.\\
			So $\sum_{i=1}^{n}\bbp(A_i) = 1.$ \\
			Morevoer, $(\cup_{k=1}^n A_k)^c  = \cap_{k=1}^n A_k^c \subset A_i^c $.\\
            Then $(\cup_{k=1}^n A_k)^c$ is finite, and $\bbp(\cup_{k=1}^n A_k) = 1 = \sum_{i=1}^{n}\bbp(A_i).$ 
			\end{enumerate}
        Thus, $\bbp$ is additive on $\llf_0$ when $\Omega$ is countably infinite.
		\item
		Consider countable many subsets $\{A_k\}_{k\geq 1}$ of $\llf_0$.\\
			If $A_k$ is finite for all $k \geq 1$, then $\bbp(A_k) = 0$. \\
			So $\sum_{k=1}^{\infty} \bbp (A_k) = 0$. \\
				However, $(\cup_{k=1}^{\infty} A_k)^c$ can be finite. \\
				For example, let $\Omega = \bbn$ and $A_k = k$ for $k \geq 1$, then $A_k$\rq s are disjoint,  and $(\cup_{k=1}^{\infty}{A_k})^c = ({ \cup_{k=1}^{\infty} }k)^c =(\bbn)^c = \emptyset$. So $\cup_{k=1}^{\infty}{A_k} \in \llf_o$, but $\bbp(\cup_{k=1}^{\infty}{A_k}) = 1 \neq \sum_{k=1}^{\infty} \bbp (A_k) $.\\
		Thus, $\bbp$ is not countably additive when $\Omega$ is countably infinite.
		\end{enumerate}
	\item
    	Just consider countable many subsets $\{A_k\}_{k\geq 1}$ of $\llf_0$.
		\begin{enumerate}[(1)]
			\item If $A_k$ is finite for all $k \geq 1$ , $\cup_{k=1}^{\infty} A_k$ is countably infinite.\\
				Then $\bbp(A_k) = 0, k \geq 1$.\\
				So $\sum_{k=1}^n \bbp (A_k) = 0$.\\
				Besides, $(\cup_{k=1}^{\infty} A_k)^c = \Omega\setminus(\cup_{k=1}^{\infty} A_k)$ is uncountably infinite since $\Omega$ is uncountable.\\
				So both $\cup_{k=1}^{\infty} A_k$ and $(\cup_{k=1}^{\infty} A_k)$ are infinite.\\
				As a result, when $A_k$ is finite for all $k \geq 1$, $\cup_{k=1}^{\infty} A_k \not\in \llf_0$. 
			\item 
			Assume there exists the unique $i \in \bbn, i \geq 1$ such that $A_i^c$ is finite.\\
            Then $\bbp(A_i) = 1$ and $\bbp(A_j) = 0$ for $j \neq i, j \geq 1$.\\
            So $\sum_{i=1}^{\infty}\bbp(A_i) = 1.$ \\
			Morevoer, $(\cup_{k=1}^{\infty} A_k)^c  = \cap_{k=1}^{\infty} A_k^c \subset A_i^c $.\\
            Then $(\cup_{k=1}^{\infty} A_k)^c$ is finite, and $\bbp(\cup_{k=1}^{\infty} A_k) = 1 = \sum_{i=1}^{\infty}\bbp(A_i).$ 
		\end{enumerate}
        Thus, $\bbp$ is countably additive on $\llf_0$ when $\Omega$ is uncountable.
	\end{enumerate}
	\end{proof}


\end{document}
