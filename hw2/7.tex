\documentclass{article}
\usepackage[english]{babel}
\usepackage{amsfonts}
%\usepackage{a4wide}
\usepackage{setspace}
\usepackage{amsmath}
\usepackage{amsfonts}
\usepackage{amssymb,enumerate}
\usepackage{amsthm,stmaryrd}
\usepackage[all]{xy}
\usepackage{hyperref}
\usepackage{flexisym}

\newcommand{\lls}{\mathcal{S}}
\newcommand{\bbp}{\mathbb{P}}
\newcommand{\bbn}{\mathbb{N}}
\newcommand{\lld}{\mathcal{D}}
\newcommand{\lla}{\mathcal{A}}
\newcommand{\llc}{\mathcal{C}}
\newcommand{\llf}{\mathcal{F}}
\newcommand{\llb}{\mathcal{B}}

\textwidth 6.0in \oddsidemargin 0.0in

\begin{document}

\vspace{3 mm}


\noindent \textbf{Problem 7} (Resnick, page 63 Problem 4) Suppose $\mathbb{P}$ is a probability measure on a $\sigma$-algebra $\mathcal{B}$ and suppose $A \notin \mathcal{B}$.  Let $\mathcal{B}_{1} := \sigma(\mathcal{B} \cup \{A\})$ and show that $\mathbb{P}$ has an extension to a probability measure $\mathbb{P}_{1}$ on $\mathcal{B}_{1}$ (Do this without applying an extension theorem).
\end{document}
