\documentclass{article}
\usepackage[english]{babel}
\usepackage{amsfonts}
%\usepackage{a4wide}
\usepackage{setspace}
\usepackage{amsmath}
\usepackage{amsfonts}
\usepackage{amssymb,enumerate}
\usepackage{amsthm,stmaryrd}
\usepackage[all]{xy}
\usepackage{hyperref}

\newcommand{\llc}{\mathcal{C}}
\newcommand{\llb}{\mathcal{B}}
\newcommand{\bbr}{\mathbb{R}}
\newcommand{\bbc}{\mathbb{C}}
\newcommand{\bbz}{\mathbb{Z}}
\newcommand{\bbq}{\mathbb{Q}}
\newcommand{\bbn}{\mathbb{N}}
\newcommand{\be}{\mathbf{e}}
\newcommand{\ba}{\mathbf{a}}
\newcommand{\fm}{\mathfrak{m}}
\newcommand{\Hom}{\operatorname{Hom}}
\renewcommand{\ker}{\operatorname{Ker}}
\newcommand{\im}{\operatorname{Im}}
\newcommand{\xra}{\xrightarrow}
\newcommand{\wti}{\widetilde}


\textwidth 6.0in \oddsidemargin 0.0in

\begin{document}

\begin{center}

\textbf{Homework 3, MATH 9010}

Due on Thursday, September 15\\

Shuai Wei

\end{center}

\vspace{3 mm}

\noindent \textbf{Problem 1} Let $(\Omega, \llb, P) = \Big(\left(\left.0,1\right]\right.,\llb\big((\left. 0,1\right]\big),\lambda\Big)$ where $\lambda$ is Lebesgue measure. Define

\begin{align*}
	X_1(\omega) &= 0, \forall \omega \in \Omega,\\
	X_2(\omega) &= 1_{\{1/2\}}(\omega),\\
	X_3(\omega) &= 1_{\bbq}(\omega)
\end{align*}
where $\bbq \in \left(\left.0,1\right]\right.$ are rational numbers in $\left(\left.0,1\right]\right.$. Note 
$$ P[X_1 = X_2 = X_3 = 0] = 1$$
and give 
$$ \sigma(X_i),~~~~	i = 1,2,3.$$

\begin{enumerate}
	\item
		$X_1(\omega) = 0, \forall \omega \in \Omega$.\\
		Note $X_1$ has range  0 , then $X^{-1}_{1}(\{0\}) = \emptyset$.\\
		So $\sigma(X_1) = \sigma(\emptyset, \Omega)= \{\emptyset, \Omega\}$.\\
	\item 
		$X_2(\omega) = 1_{\{1/2\}}(\omega)$.\\
	Note $X_2$ has range $\{0,1\}$, then $X^{-1}_1(\{0\}) = \{(0,1/2)\cup (1/2,1]\}$,$X^{-1}_2(\{1\}) = \{1/2\}$. \\
	So $\sigma(X_2) = \{(0,1/2)\cup (1/2,1], 1/2, \emptyset, \Omega\}$.
	\item
	$X_3(\omega) = 1_{\bbq}(\omega)$.\\
	Note, $X_3$ has range $\{0,1\}$, then $X^{-1}_1(\{0\}) = \{\bbq\}$, $X^{-1}_1(\{1\}) = \{\bbq^c \cap (0,1]\}$.\\
	So $\sigma(X_3) = \{\bbq, \bbq^c \cap (0,1], \emptyset, \Omega\}$.\\
	where $\bbq \in \left(\left.0,1\right]\right.$ are rational numbers in $\left(\left.0,1\right]\right.$.
\end{enumerate}


\end{document}
