\documentclass{article}
\usepackage[english]{babel}
\usepackage{amsfonts}
%\usepackage{a4wide}
\usepackage{setspace}
\usepackage{amsmath}
\usepackage{amsfonts}
\usepackage{amssymb,enumerate}
\usepackage{amsthm,stmaryrd}
\usepackage[all]{xy}
\usepackage{hyperref}

\newcommand{\llc}{\mathcal{C}}
\newcommand{\llb}{\mathcal{B}}
\newcommand{\bbr}{\mathbb{R}}
\newcommand{\bbc}{\mathbb{C}}
\newcommand{\bbz}{\mathbb{Z}}
\newcommand{\bbq}{\mathbb{Q}}
\newcommand{\bbn}{\mathbb{N}}
\newcommand{\be}{\mathbf{e}}
\newcommand{\ba}{\mathbf{a}}
\newcommand{\fm}{\mathfrak{m}}
\newcommand{\Hom}{\operatorname{Hom}}
\renewcommand{\ker}{\operatorname{Ker}}
\newcommand{\im}{\operatorname{Im}}
\newcommand{\xra}{\xrightarrow}
\newcommand{\wti}{\widetilde}


\textwidth 6.0in \oddsidemargin 0.0in
\setcounter{page}{5}
\begin{document}

\noindent \textbf{Problem 5} Suppose $\infty < a \leq b < \infty$. Show that the indicator function $1_{\left.\left(a,b\right.\right]}(x)$ can be approximated by bounded and continuous functions; that is, show that there exist a sequence of contious functions $0 \leq f_n \leq 1$ such that $f_n \to 1_{\left.\left(a,b\right.\right]}$ pointwise.

\begin{proof}
	$ $ \newline
	Let
	\begin{align*}
		f_n(x) &= n(x-a)1_{\left.\left(a,a+1/n\right.\right]}(x) + 1_{\left.\left(a+1/n,b\right.\right]}(x) - n(x-b-1/n) \left.\left(b,b+1/n\right.\right](x)\\
			&= 
			\begin{cases}
		      	n(x-a), & \text{if}\ x \in \left.\left(a,a+1/n\right.\right],\\
		      	1, & \text{if}\ x \in \left.\left(a+1/n,b\right.\right],\\
		      	- n(x-b-1/n), & \text{if}\ x \in \left.\left(b,b+1/n\right.\right].
		    \end{cases}
	\end{align*}
	Then we can find $0 \leq f_n \leq 1$ given $\infty < a \leq b < \infty$.\\
	$f_n((a+1/n)^{-}) = 1 = f_n((a+1/n)^{+}) = 1 = f_n(a+1/n)$ and $f_n(b^{-}) = 1 = f_n(b^{+}) = 1 = f_n(b)$.\\
	Namely, $f$ is continuous at point $a+1/n$ and $b$.\\
	It is obvious that $f$ is also continuous on other points of $\left.\left(a,b\right.\right]$.\\
	Thus, $f_n$ is continuous on $\left.\left(a,b\right.\right]$.\\
	Let $x_0 \in \left.\left(a,b\right.\right] $. Then there exists $N \in \bbn$ such that $a+1/n < x_0 \leq b$ as $n \geq N$ .\\	
	So when $n > N$, $f_n(x_0) = 1$ since $x_0 \in \left.\left(a+1/n,b\right.\right]$ as $n \geq N$. \\
	So $|f_n(x_0) - 1_{\left.\left(a,b\right.\right]}(x_0)| = 0$ as $n \geq N$.\\
	Hence, $f_n \to 1_{\left.\left(a,b\right.\right]}$ pointwise.

\end{proof}


\end{document}
