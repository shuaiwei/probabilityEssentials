\documentclass{article}
\usepackage[english]{babel}
\usepackage{amsfonts}
%\usepackage{a4wide}
\usepackage{setspace}
\usepackage{amsmath}
\usepackage{amsfonts}
\usepackage{amssymb,enumerate}
\usepackage{amsthm,stmaryrd}
\usepackage[all]{xy}
\usepackage{hyperref}
\usepackage{bbold}

\newcommand{\idca}{\mathbb{1}}
\newcommand{\llll}{\mathcal{L}} 
\newcommand{\llc}{\mathcal{C}}
\newcommand{\llb}{\mathcal{B}}
\newcommand{\bbr}{\mathbb{R}}
\newcommand{\bbc}{\mathbb{C}}
\newcommand{\bbz}{\mathbb{Z}}
\newcommand{\bbq}{\mathbb{Q}}
\newcommand{\bbn}{\mathbb{N}}
\newcommand{\be}{\mathbf{e}}
\newcommand{\ba}{\mathbf{a}}
\newcommand{\fm}{\mathfrak{m}}
\newcommand{\Hom}{\operatorname{Hom}}
\renewcommand{\ker}{\operatorname{Ker}}
\newcommand{\im}{\operatorname{Im}}
\newcommand{\xra}{\xrightarrow}
\newcommand{\wti}{\widetilde}


\textwidth 6.0in \oddsidemargin 0.0in

\begin{document}

\begin{center}

\textbf{Homework 4, MATH 9010}

Due on Thursday, September 15\\

Shuai Wei

\end{center}

\vspace{3 mm}

\noindent \textbf{Problem 1} Let $X\in \llll^1$ and let $A_n$ be a sequence of events such that $\lim_{n\to \infty}P(A_n) = 0$. Show that $\lim_{n\to \infty}E\{X\idca_{A_n}\}= 0$. \\

\begin{proof}
Since $X$ is integrable, $|X|$ is also integrable.\\ 
Then 
	\[E(|X|)  < \infty.\]
Let $n \in \bbn$, by Markov inequality, 	
\[P(|X|=\infty) \leq  P(|X|>n) \leq \frac{E(|X|)}{n} \]	
Since 
\[\lim_{n \to \infty}\frac{E(|X|)}{n} = 0,\]
we have 
\[P(|X|=\infty) = 0.\]
So 
\[E(|X|\idca_{\{|X| = \infty\}})= 0\]
by what we have shown in class.\\
Besides,
\begin{align*}
  E(|X|\idca_{A_n}) &= E(|X|\idca_{A_n\{|X| = \infty\}}) + E(|X|\idca_{A_n\{|X| < \infty\}}) \\
			    &\leq E(|X|\idca_{\{|X| = \infty\}}) + E(|X|\idca_{A_n\{|X| < \infty\}}) \\
  		        &=E(|X|\idca_{A_n\{|X| < \infty\}})
\end{align*}

There exist $M > 0$ such that
\[ |X|\idca_{A_n\{|X| < \infty\}} \leq M\idca_{A_n\{|X| < \infty\}} = M\idca_{A_n}. \]
So 
\[ E(|X|\idca_{A_n}) \leq E(|X|\idca_{A_n\{|X| < \infty\}}) \leq  M P(A_n). \]
Then
\[\lim_{n\to \infty}E\{|X|\idca_{A_n}\} \leq M \lim_{n\to \infty} P(A_n) = 0.\]
Thus 
\[\lim_{n\to \infty}E\{|X|\idca_{A_n}\} =0 .\]
Since 
\[\lim_{n\to \infty}E\{|X|\idca_{A_n}\} = \lim_{n\to \infty}E\{X^{+}\idca_{A_n}\} + \lim_{n\to \infty}E\{X^{-}\idca_{A_n}\}  = 0, \]
	we have 
	\[\lim_{n\to \infty}E\{X^{+}\idca_{A_n}\} = \lim_{n\to \infty}E\{X^{-}\idca_{A_n}\}  = 0\]
Therefore,
 \[\lim_{n\to \infty}E\{X\idca_{A_n}\} = \lim_{n\to \infty}E\{X^{+}\idca_{A_n}\} - \lim_{n\to \infty}E\{X^{-}\idca_{A_n}\}  = 0 \]

\end{proof}

\newpage

\noindent \textbf{Problem 2} Suppose $X_n, n \geq 1$ and $X$ are uniformly bounded random; i.e. there exists a constant $K$ such that 
\[ |X_n| \vee |X| \leq K.\]
If $X_n \to X$ as $n \to \infty $, show by means of dominated convergence that 
\[E|X_n-X| \to 0.\]
\begin{proof}
  For any $n \in \bbn$,
  \[|X_n - X| \leq |X_n| + |X| \leq M + M =2M, \]
  and \\
  Constant $M$ is integrable since 
  \[ E(M)= E(M\idca_{\Omega}) = M\times P({\Omega}) =M < \infty.\]
  Since $X_n \to X$, $X_n - X \to 0$.\\
  By Dominated Convergence Theorem,
  \[\lim_{n\to \infty} E(X_n - X) = E(0) = 0.\]
  Namely,
  \[E(X_n - X) \to 0.\]

\end{proof}
\newpage

\noindent \textbf{Problem 3} For $X \geq 0$, let 
\[X_n^* = \sum_{k=1}^{\infty}\frac{k}{2^n}\idca_{[\frac{k-1}{2^n} \leq X < \frac{k}{2^n}]}.\]
Show 
\[ E(X_n^*) \downarrow E(X).\]

\begin{proof}
  We first show $\{X_n^*\}_{n \geq 1}$ is monotone decreasing.\\
  For $k,n \in \bbn$ and $k,n\geq 1$, 
  \begin{align*}
  	 \left.\left[\frac{k-1}{2^n},\frac{k}{2^n}\right.\right)  &= \left.\left[\frac{k-1}{2^n}, \frac{k-1/2}{2^n}\right)\right. \bigcup \left.\left[\frac{k-1/2}{2^n}, \frac{k}{2^n}\right)\right. \\
   															 &= \left.\left[\frac{2k-2}{2^{n+1}}, \frac{2k-1}{2^{n+1}}\right)\right. \bigcup \left.\left[\frac{2k-1}{2^{n+1}}, \frac{2k}{2^{n+1}}\right)\right.  
 \end{align*}
   	So 
   	\[ X_n^* = \sum_{k=1}^{\infty}\frac{2k}{2^{n+1}}\idca_{[\frac{2k-2}{2^{n+1}} \leq X < \frac{2k-1}{2^{n+1}}]} + \frac{2k}{2^{n+1}}\idca_{[\frac{2k-1}{2^{n+1}} \leq X < \frac{2k}{2^{n+1}}]}.\]
  On the other hand,
  \[ X_{n+1}^* = \sum_{k=1}^{\infty}\frac{2k-1}{2^{n+1}}\idca_{[\frac{2k-2}{2^{n+1}} \leq X < \frac{2k-1}{2^{n+1}}]} + \frac{2k}{2^{n+1}}\idca_{[\frac{2k-1}{2^{n+1}} \leq X < \frac{2k}{2^{n+1}}]}. \]
Thus,
	$X_n^* \geq X_{n+1}^*$. \\
We claim $X_n^* \downarrow X$.\\
For any $\epsilon > 0$, there exists $N \in \bbz$ such that $\frac{1}{2^N} < \epsilon$.\\
Then by the definition of $X_N^*$, $X_N^* \geq X$ and so  
\[|X_N^*-X| = X_N^* -X < \frac{1}{2^N} < \epsilon. \]
So when $n > N$, $X_n < X_N $ and 
\[|X_N^*-X| =X_n^* -X < X_N^* -X < \epsilon.\]
Therefore, $X_n^* \downarrow X$.\\
Let $Y_n^* = X_1^*-X_n^* \geq 0$, then $Y^*_n$ is monotone increasing.\\
Then 
\[Y_n^* \uparrow X_1^*-X.\]
By the Monotone Convergence Theorem, we have
\[E(Y_n^*) \uparrow E(X_1^*-X).\]
Then 
\[E(X_1^*)-E(X_n^*) \uparrow E(X_1^*)-E(X).\]
since $E(Y_n^*)=E(X_1^*)-E(X_n^*)$.\\
As a result, 
\[E(X_n^*) \downarrow E(X).\]

\end{proof}

\noindent \textbf{Problem 4} Suppose $X$ is a non-negative random variable satisfying 
\[ P[0 \leq X < \infty] = 1.\]
Show
\begin{enumerate}[(a)]
	\item 
	  \[ \lim_{n \to \infty } n E\Big( \frac{1}{X}\idca_{[X>n]} \Big)  = 0,\]
	\item
	  \[ \lim_{n \to \infty } n^{-1} E\Big( \frac{1}{X}\idca_{[X>n^{-1}]} \Big)  = 0.\]
\end{enumerate}

\begin{enumerate}[(a)]
	\item 	
		\begin{proof}
			\begin{align*}
			  0\leq nE\Big( \frac{1}{X}\idca_{[X>n]}\Big) &= n \int_{\idca_{[X>n]}} \frac{1}{X(\omega)}dP(\omega) \\
		 					  				  			  &= \int_{\idca_{[X>n]}} \frac{n}{X(\omega)}dP(\omega) \\
			  										      & \leq \int_{\idca_{[X>n]}}dP(\omega) \\
			  										      &= P(\idca_{[X>n]})
			\end{align*}
		\end{proof}
		So 
		\[0\leq \lim_{n \to \infty} nE\Big( \frac{1}{X}\idca_{[X>n]}\Big) \leq \lim_{n \to \infty} P(\idca_{[X>n]}) = P[X=\infty] = 1- P[0 \leq X < \infty]  =0.\]
		Thus 
		\[\lim_{n \to \infty} nE\Big( \frac{1}{X}\idca_{[X>n]}\Big) = 0. \]
	\item
	  \begin{proof}
	  	$ $\newline
		When $X = 0, \idca_{[X>n^{-1}]} = 0$, then $\frac{1}{X} \idca_{[X>n^{-1}]} = 0$.\\
	  	So $n^{-1}E\big(\frac{1}{X} \idca_{[X>n^{-1}]}\big) = 0$. \\
		When $X \neq 0,~ P(X=0) = 0$.
		\begin{align*}
		 0\leq  n^{-1}E\Big(\frac{1}{X} \idca_{[X>n^{-1}]}\Big) &= n^{-1}E\Big(\frac{1}{X} \idca_{[\frac{1}{X}<n]}\Big) \\
		  												& \leq n^{-1}E\Big(n \idca_{[\frac{1}{X}<n]}\Big)\\
		  												&= E\Big(\idca_{[\frac{1}{X}<n]}\Big) \\
		  												&= P(\frac{1}{X}<n) 
		\end{align*}
		So 
	  \[0\leq \lim_{n \to \infty}n^{-1}E\Big(\frac{1}{X} \idca_{[X>n^{-1}]}\Big) \leq \lim_{n \to \infty} P(\frac{1}{X}<n) = P(X=0) = 0. \]
		Thus,
		\[\lim_{n \to \infty } n^{-1} E\Big( \frac{1}{X}\idca_{[X>n^{-1}]} \Big)  = 0.\]
	  \end{proof}
\end{enumerate}


\newpage

\noindent \textbf{Problem 5} Suppose $\{p_k,k \geq 0\}$ is a probability mass function on $\{0,1,...\}$ and define the generating function 
\[ P(s) = \sum_{k=0}^{\infty}p_ks^k, ~~0 \leq s \leq 1.\]
Prove using dominated convergence that 
\[ \frac{d}{ds}P(s) = \sum_{k=1}^{\infty}p_kks^{k-1},~~0\leq s \leq 1, 	\]
that is, prove differentiation and summation can be interchanged.\\

\begin{proof}
  Let $X$ be the corresponding random variable. Then $P(X = k) = p_k$.

  \begin{align*}
  	\frac{d}{ds}P(s) &= \frac{d}{ds}E\big(s^{X}\big) \\
  					 &= \lim_{t \to s} \frac{E\big(s^{X}\big) - E\big(t^{X}\big)}{s-t} \\
  					 &= \lim_{t \to s} \frac{E\big(s^{X} - t^{X}\big)}{s-t} \\
  					 &= \lim_{t \to s} E\left(\frac{s^{X} - t^{X}}{s-t}\right). \\
  \end{align*}
  Then we choose a sequence of numbers $\{t_n\}_{n=1}^{\infty}$ such that $t_n \downarrow s$ and $t_n \neq s$.
  So 
  \begin{align*}
	\frac{d}{ds}P(s) &= \lim_{t_n \to s} E\left(\frac{s^{X} - t_n^{X}}{s-t_n}\right) \\
					&=\lim_{t_n \to s} E(Y_n) 
  \end{align*}
  by letting $Y_n = \frac{s^{X} - t_n^{X}}{s-t_n}$.\\
  Then 	
  \begin{align*}
  	 \lim_{n \to \infty}Y_n &=\lim_{n \to \infty} \frac{s^{X} - t_n^{X}}{s-t_n}\\
  							&= \lim_{n \to \infty} \frac{- (X-1)t_n^{X}}{-1}\\ 
  							&= \lim_{n \to \infty} (X-1)t_n^{X} \\
  							&= (X-1)s^X.
  \end{align*}
  Let $f(y) = y^X$, then $f'(y) = Xy^{X-1}$.\\
  By Mean Value Theorem, there exists $\theta \in [t_n,~s]$ such that $f(s) -f(t_n) = (s-t_n)(f'(\theta))$.\\
  So 

  \begin{align*}
  	\left| Y_n\right| = \left |\frac{s^X -t_n^X}{s-t_n} \right |  &= \left|f'(\theta)\right| \\
  											&= \left|X \theta^{X -1}\right| \\
  											& \leq  Xs^{X-1}.  
  \end{align*}

 Since $Xs^{X-1}$ is an integrable random variable, by Dominated Convergence Theorem,
 \[\lim_{t_n \to s} E(Y_n) = E(\lim_{n \to \infty}Y_n) = E\left((X-1)s^X\right)\]
 Since $\lim_{t_n \to s} E(Y_n) = \frac{d}{ds}P(s)$ and $E\left((X-1)s^X\right) = \sum_{k=1}^{\infty}p_kks^{k-1}$,
 we have
 \[ \frac{d}{ds}P(s) = \sum_{k=1}^{\infty}p_kks^{k-1}.\]
\end{proof}







\end{document}

